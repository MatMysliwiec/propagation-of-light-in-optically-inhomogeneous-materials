Metoda Bogackiego-Shampine’a jest bardziej wydajną oraz ma lepszą stabilność aniżeli inne tego typu metody (np. Dormand-Prince’a, Fehlberga). Opartą ją na trójstopniowej formule trzeciego rzędu Ralstona, ze względu na najlepszą stabilność. Ponadto, dodano automatyczną kontrolę wielkości kroku za pomocą metody drugiego rzędu, co wiąże się z niewielkim kosztem na krok. Nie tylko kontrola zwraca się poprzez zapewnienie najbardziej efektywnego rozmiaru kroku, ale także unika rozmiarów kroków, które prowadzą do niestabilności.

Do utrzymania niskich kosztów na krok, ograniczono się do minimalnej liczby etapów. Przy doborze parametrów sugerowano się, aby były w pewnym sensie małe, tak aby mieć formułę dokładną dla ,,typowego"~problemu \cite{bogacki19893}. Najlepszą możliwością jest uczynienie niektórych z~tych współczynników zerowymi, ponieważ wtedy dla niektórych problemów wzór będzie rzędu 3,~co można było uzyskać przy pomocy wzorów przedstawionych przez Ralstona \cite{ralston1962runge}:
$$k_{1}=h_{n}f\br{x_{n},y_{n}},$$
$$k_{2}=h_{n}f\br{x_{n}+\alpha h_{n},y_{n}+ A \alpha k_1},$$
$$k_{3}=h_{n}f\br{x_{n}+\beta h_{n},y_{n}+ B \gamma  k_{1} + B (\beta - \gamma) k_{2}},$$
\begin{equation}\label{eq:RK3}
y_{n+1}-y_{n}=w_{1} k_{1}+w_{2} k_{2}+(1 - w_{1} -w_{2}) k_{3}.
\end{equation}
Po wyznaczeniu parametrów dowolnych, zakładając $A=1, \quad B=1$, formuła będzie wyglądała następująco:
$$k_{1}=h_{n}f\br{x_{n},y_{n}},$$
$$k_{2}=h_{n}f\br{x_{n}+(\frac{1}{2})h_{n},y_{n}+\frac{1}{2}k_1},$$
$$k_{3}=h_{n}f\br{x_{n}+(\frac{3}{4})h_{n},y_{n}+(\frac{3}{4})k_{2}},$$
$$y_{n+1}-y_{n}=(\frac{2}{9})k_{1}+(\frac{1}{3})k_{2}+(\frac{4}{9})k_{3}.$$
Wzór ten nie do końca minimalizuje normę euklidesową współczynników błędu obcięcia, ale różnica jest zbyt mała, aby miała jakiekolwiek znaczenie praktyczne.

\section{Metoda Bogackiego-Shampine'a RK3(2)}\label{s:A}
\indent Wykorzystano dwustopniowe wzory RK, aby ponownie wykorzystać etapy powstałe w ocenie formuły rzędu trzeciego, aby otrzymać wynik $y_{n+1}$ rzędu drugiego. Wszystkie wzory drugiego rzędu Rungego - Kutta mają ten sam obszar stabilności, podobnie jak wszystkie trzystopniowe wzory trzeciego rzędu. W przypadku dwustopniowego istnieje niewielka elastyczność w zakresie uzyskania wysokiej jakości oszacowania błędu. Dodatkową elastyczność uzyskuje się poprzez uświadomienie sobie, że większość kroków kończy się sukcesem, a jeśli krok jest sukcesem, pierwszym etapem następnego kroku jest zawsze $f(x_{n+1},\hat{y}_{n+1})$. Poprzez dodanie etapu 
$$k_{4}=f\br{x_{n}+h,\hat{y}_{n}+h\sum_{j=1}^{3}\hat{b}_{j}k_{i}},$$
i rozważenie formuły drugiego rzędu w postaci:
$$y_{n+1}=\hat{y}_{n}+h\sum_{i=1}^{4}b_{i}k_{i},$$
uzyskuje się elastyczność przy bardzo niewielkim dodatkowym koszcie. To podejście jest nazywane FSAL, First Same As Last. W tym podejściu określamy, która z pary formuł zostanie użyta do posunięcia kroku naprzód, czyli lokalną ekstrapolacje \cite{bogacki19893}. 

Opracowanie formuły Bogackiego-Shampine'a RK3(2) przy pomocy narzędzi komputerowej algebry komputerowej Mathematica\footnote{Przygotowany na podstawie materiałów do wykładu z metod obliczeniowych dla studentów kierunku
Fizyka Techniczna PK}. Pracujemy na formule trzeciego rzędu Ralstona, ale żeby zmniejszyć do minimum błąd metody oraz użyć później tego parametru do kontroli kroku przy całkowaniu toru ruchu promienia świetlnego używamy go wraz z formułą drugiego rzędu. Parametry do formuły trzeciego rzędu uzyskaliśmy przyjmując założenia:
\begin{enumerate}
\item symetria przyrostów punktów węzłowych, tzn, jeśli x$\rightarrow$ x + A $\Delta$x to y$\rightarrow$ y + A $\Delta$ y
\item wagi $w1,~w2,~w3 = 1-w1-w2$ sumują się do 1
\end{enumerate}
Jak zostało wspomniane dodatkowo przyjęliśmy, że $A=1, \quad B=1$, mając wszystkie założenia tworzymy formułę RK trzeciego rzędu, została wypisana powyżej \ref{eq:RK3}. Sama procedura wyglądała podobnie jak teraz prezentowane omówienie tworzenia metody RK(3)2.
\begin{lstlisting}
    symb[expr_]:= expr /. {Derivative[i_,j_][F][x,Y[x]] 
        $\rightarrow$ ToExpression["F" <> ToString[i] <> ToString[j]], 
        F[x,Y[x]] $\rightarrow$ F}
\end{lstlisting}
Funkcja skraca symbolikę, aby późniejsze wyniki były czytelniejsze. Zmienia zapis pochodnych cząstkowych poprzez przekształcenie ich w postać string, dzięki czemu zamiast $F^{(0,1)}[x,Y[x]]$ dostajemy $F01[x,Y[x]]$. Dodatkowo wyrzuca niepotrzebny symbol argumentów $[x,Y[x]].$ Zamiast \begin{lstlisting} 
    Y$^{''}$[x] $\rightarrow$ Y$^{'}$[x] F$^{(0,1)}$[x,Y[x]] + F$^{(1,0)}$[x,Y[x]], \end{lstlisting} 
dostajemy 
\begin{lstlisting} 
    Y$^{''}$[x] $\rightarrow$ F F01 + F10.\end{lstlisting} Cyfry przy funkcji opisujących układ za pomocą związków położeń z prędkościami F, oznaczają stopień pochodnej cząstkowej odpowiednio po pierwszej zmiennej i drugiej zmiennej.
\begin{lstlisting}
    repl = Nest[Flatten@ {Collect[D[First @#, x], 
        F[x,Y[x]], Simplify] //. #,#} &, 
        {Y$^{'}$[x] $\rightarrow$ F[x,Y[x]]}, 3];
    Print[Reverse@% // symb // TableForm]
\end{lstlisting}
Każda kolejna reguła odwołuje się do wielkości $Y'$, $Y''$ itd., więc należy dokonać odpowiednich podstawień, używamy do tego zmiennej $repl$, która działa na podstawie analitycznych informacji o $F$ definiującej układ różniczkowy $dY = F[x, Y]$, $Y$ jest wektorem o dowolnym wymiarze, a~$F$ jest funkcja wektorowa zależna od $x$ i $Y$, jak było to wcześniej opisane w \ref{s:spro_ukl}.
\begin{lstlisting}
    Out:
    Y$^{'}$[x] $\rightarrow$ F
    Y$^{''}$[x] $\rightarrow$ F F01 + F10
    Y(3)[x] $\rightarrow$ F01 F10 + F (F01$^2$ + F11) + F (F F02 + F11) + F20
    Y(4)[x] $\rightarrow$ F01$^2$ F10 + 3 F10 F11 + F$^2$ (F F03 + F12) + F01 F20 + 
        + F (F01$^3$ + F02 F10 + 3 F01 F11 + F21) + 2 F (F02 F10 + 
        + F01 F11 + F (2 F01 F02 + F12) + F21) + F30
\end{lstlisting}
Dostajemy rozwinięcia szeregu Taylora do rzędu czwartego, które później posłużą nam do zapisania formalnego szeregu Taylora dla integratora. 
\begin{lstlisting}
    Series[Y[x+h], {h,0,4}] //. repl;
    YE = % // symb;
    Print[Y[x+h],"==",YE]
\end{lstlisting}
Wykorzystujemy powyższe wyniki do zapisania formalnego szeregu Taylora, który definiuje nam wzorzec, do którego integrator powinien dążyć, oznaczmy ten wzorzec jako YE = Yexact. Wykorzystujemy do tego już definiowaną formułę $repl$ oraz $symb$ do czytelniejszego zapisu. 
\begin{lstlisting}
    Out:
    Y[h+x] == Y[x] + F h + $\frac{1}{2}$ (F F01 + F10) h$^2$ +
        + $\frac{1}{6}$ (F01 F10 + F (F01$^2$ + F11) + F (F F02 + F11) +
        + F20) h$^3$ + $\frac{1}{24}$ (F01$^2$ F10 + 3 F10 F11 + F$^2$ (F F03 + F12) +
        + F01 F20 + F (F01$^3$ + F02 F10 + 3 F01 F11 + F21) +
        + 2 F (F02 F10 + F01 F11 + F (2 F01 F02 + F12) + F21) +
        + F30) h$^4$ + O[h]$^5$
\end{lstlisting}
Definiujemy formułę trzeciego rzędu Ralstona, której parametry zostały już uzyskane osobno,~poprzez jak wspomniałem podobne kroki z założeniami napisanymi powyżej.
\begin{lstlisting}
    k1 = h F[x,Y[x]];
    k2 = h F[x + $\frac{1}{2}$h,Y[x] + $\frac{1}{2}$k1];
    k3 = h F[x + $\frac{3}{4}$h,Y[x] + 0k1 + $\frac{3}{4}$k2];
    Y3 = Y[x] + $\frac{2}{9}$k1 + $\frac{1}{3}$k2 + $\br{1-\frac{2}{9}-\frac{1}{3}}$k3;
\end{lstlisting}
Konstruujemy kombinacje dającą drugi rząd, którą będziemy używać razem z powyższą formułą do kontroli błędu podczas całkowania toru ruchu promienia świetlnego, co można zauważyć przy $Y2$, która jest kombinacją dającą drugi rząd jednak z błędem $h^4$:
\begin{lstlisting}
    k4 = h F[x+h,Y3];
    Y2 = Y[x] + b1k1 + b2k2 + b3k3 + (1-b1-b2-b3)k4;
\end{lstlisting}
Poniżej konstruujemy odpowiednik metody trójpunktowej, a więc z punktem pośrednim. Niech $x=a$,\quad $x+h=b$, natomiast punkt pośredni to pewne $c$~: $a<c<b$. Reguła Eulera byłaby nastepująca: $Yb=Ya+h F[c,Yc] $. Problemem tu jest nieznany parametr $c$. Można się jednak posłużyć kombinacją typu
$Yb = Ya + b1~h F[a,Ya]+ b2~h F[a,Ya] + b3~F[c,Y] + (1 - b1 - b2 - b3)~h F[a,Ya]]$,~czyli mamy pewna średnia wagową z parametrami $b1, b2, b3, c$  do wyznaczenia przy pomocy jakiegoś optymalizującego kryterium. Uzyskujemy aproksymacje ,,trójpunktową"~YA3 do dokładności do trzeciego rzędu oraz aproksymację YA2, czyli oczekujemy dokładności 2-go rzędu.  Wybieramy dowolne współczynniki tak, by te rzędy zapewnić:
\begin{lstlisting}
    YA2 = Series[Y2,{h, 0, 3}] // symb // Simplify;
    Print[Y[x+h],"==", YA2]
    YA3 = Series[Y3,{h, 0, 4}] // symb // Simplify;
    Print[Y[x+h],"==", YA3]
\end{lstlisting}
\begin{lstlisting}
    Out:
    Y[h+x] == Y[x] + F h - $\frac{1}{4}$((-4 + 4b1 + 2b2 + b3) (F F01 +
        + F10)) h$^2$ +  $\frac{1}{32}$(4b2 (F$^2$ F02 + 2 F F11 + F20) -
        - 16(-1 + b1 + b2 + b3) (F$^2$ F02 + F01 F10 +
        + F (F01$^2$ + 2 F11) + F20) + 3b3(4 F F01$^2$ + 3 F$^2$ F02 + 
        + 4 F01 F10 + 6 F F11 + 3 F20)) h$^3$ + O[h]$^4$
    
    Y[h+x] == Y[x] + F h + $\frac{1}{2}$(F F01 + F10) h$^2$ +
        + $\frac{1}{6}$(F$^2$ F02 + F01 F10 + F (F01$^2$ + 2 F11) + F20) h$^3$ +
        + $\frac{1}{288}$(11 F$^3$ F03 + 36 F10 F11 + F$^2$ (48 F01 F02 + 33 F12) +
        + 12 F01 F20 + F (36 F02 F10 + 60 F01 F11 + 33 F21) +
        + 11 F30) h$^4$ + O[h]$^5$
\end{lstlisting}
Pierwszy etap znalezienia odpowiedniego kryterium, to znaczy wyznaczenie rozwiązania dla parametru $b3$
\begin{lstlisting}
    sol = Flatten@ Solve[-2 + 4b1 + 2b2 + b3 == 0, b3]
\end{lstlisting}
\begin{lstlisting}
    Out:
    {b3$\rightarrow$-2 (-1 + 2b1 + b2)}
\end{lstlisting}
Kolejnym etapem jest porównanie odpowiednich parametrów z formuł trzeciego i drugiego rzędu. Związujemy w formule $tmp$ parametry o identycznych kombinacjach wyznaczając $b1$ oraz $b2$, aby później to wykorzystać do wyznaczenia parametrów $b$:
\begin{lstlisting}
    tmp = Flatten@ Solve[{-5 + 18b1 + 3b2 == A, 1 - 3b2 == B},
        {b1, b2}];
    sol1 = Flatten@ {sol /. tmp, tmp} // Simplify
\end{lstlisting}
\begin{lstlisting}
    Out:
    {b3$\rightarrow$ -$\frac{2}{9}$(A - 2(1 + B)), b1$\rightarrow$$\frac{1}{18}$(4 + A + B), b2$\rightarrow$$\frac{1 - B}{3}$}
\end{lstlisting}
Na koniec wyznaczenie parametrów używając wcześniejszego etapów z dodaniem parametru $\alpha$,~który jest naszym punktem środkowym w metodzie trójpunktowej, dzięki czemu zależnie od niego dostajemy rodzinę integratorów trzeciego rzędu.
\begin{lstlisting}
    sol2 = sol1 /. {A$\rightarrow$$\alpha$, B$\rightarrow$$\frac{1}{4}$$\alpha$} // Simplify
\end{lstlisting}
\begin{lstlisting}
    Out:
    {b3$\rightarrow$$\frac{4 - \alpha}{9}$, b1$\rightarrow$$\frac{2}{9}$ + $\frac{5 \alpha}{72}$, b2$\rightarrow$$\frac{4 - \alpha}{12}$}
\end{lstlisting}
Podsumowanie wyników dla $\alpha==1$:
\begin{lstlisting}
    k1 = h F[x, Y[x]];
    k2 = h F[x + $\frac{1}{2}$h, Y[x] + $\frac{1}{2}$k1];
    k3 = h F[x + $\frac{3}{4}$h, Y[x] + $\frac{3}{4}$k2];
    k4 = h F[x + h, Y[x] + $\frac{2}{9}$k1 + $\frac{1}{3}$k2 + $\frac{4}{9}$k3];
    Y[x] + $\frac{2}{9}$k1 + $\frac{1}{3}$k2 + $\frac{4}{9}$k3 + 0k4;
\end{lstlisting}

\section{Przykład zastosowania}\label{s:AA}
Realizacja procedury kroku całkującego przez integrator oparty na metodzie Bogackiego-Shampine'a RK(3)2 na przykładzie oscylatora harmonicznego: 
$$y'' + y == \frac{x}{2}\Rightarrow y1=y,\quad y2=y' \Rightarrow y1'=y2,\quad y2'=-y1+x \Rightarrow F[x_, y1_, y2_{}] = {y2, -y1 + \frac{x}{2}}$$
z warunkami początkowymi $y(0) = 0$, $y'(0) = 2$. Przez parametry $a$ i $b$ określamy długość łuku,~który oscylator musi przebyć. $h$ określamy długość kroku całkowania, którym manewrujemy dzieląc długość drogi $(b-a)$ przez stałą $n$. Rysunki wygenerowane przez program ,,Oscy\_harm".
\begin{lstlisting}
    F[x_, y1_, y2_] = {y2, -y1 + x/2};
    a = 0.; 
    b = 6$\pi$;
    y0 = 0;
    dy0 = 2;
    n = 400;
    h = (b-a)/n;
\end{lstlisting}
\begin{figure}[H]
    \centering
    \includegraphics[width=0.65\textwidth]{grafika/Rys4_compressed.pdf}
    \caption{Porównanie rozwiązania numerycznego (czerwona ciągła linia) z rozwiązaniem ścisłym (niebieska kropkowana linia). Oś rzędnych - $x$, oś odciętych - $f(x)$}
\end{figure}
\begin{figure}[H]
    \centering
    \includegraphics[width=0.65\textwidth]{grafika/Rys7_compressed.pdf}
    \caption{Wielkość kroku, jaki integrator dobiera podczas całkowania. Oś rzędnych - indeks kroku $h_i$, oś odciętych - wartość kroku}
\end{figure}
\begin{figure}[H]
    \centering
    \includegraphics[width = 0.65\textwidth]{grafika/Rys5_compressed.pdf}
    \caption{Różnice między rozwiązaniem numerycznym i ścisłym dla różnych wartości parametru $n$, który dzieli krok początkowy $h$. Niebieska linia kropkowana wartość parametru $n = 50$; zielona linia kropkowana $n = 100$; czerwona linia kropkowana $n = 200$; żółta linia kropkowana $n = 400$. Oś rzędnych - $x$, oś odciętych - różnica pomiędzy wartością rozwiązania numerycznego YA, a wartością rozwiązania ścisłego YE, YA-YE}
\end{figure}
\begin{figure}[H]
    \centering
    \includegraphics[width = 0.65\textwidth]{grafika/Rys6_compressed.pdf}
    \caption{Różnice między rozwiązaniem numerycznym i ścisłym dla różnych wartości parametru $n$, który dzieli krok początkowy $h$. Przybliżenie na linie przy dzieleniu kroku przy parametrze $n = 200$ oraz $n = 400$. Oś rzędnych - $x$, oś odciętych - różnica pomiędzy wartością rozwiązania numerycznego YA, a wartością rozwiązania ścisłego YE, YA-YE}
\end{figure}


