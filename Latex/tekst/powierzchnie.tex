\quad \quad 
Szczególny przypadek fal elektromagnetycznych, w których pole zależy tylko od jednej współrzędnej, są fale płaskie. W przypadku tym równania pola przybierają postać:
%
$$\frac{\partial^2 f}{\partial t^2}-c^2 \frac{\partial^2 f}{\partial x^2}=0.$$
%
Cechą fali płaskiej jest to, że kierunek jej rozchodzenia się oraz amplituda są wszędzie jednakowe. Własności tych oczywiście nie ma dowolna fal elektromagnetyczna. Często jednak dowolne fale elektromagnetyczne, niebędące falami płaskimi, można w małych obszarach przestrzeni uważać za fale płaskie. Ujęcie takie jest słuszne jedynie wtedy, gdy wzdłuż odległości rzędu długości fali amplituda i kierunek rozchodzenia się fali prawie się nie zmieniają. Jeśli warunek jest spełniony, można mówić o powierzchni falowej, to znaczy powierzchni, na której we wszystkich punktach faza fali jest jednakowa. Powierzchnia falowa jest oczywiście płaszczyzną prostopadłą do kierunku jej rozchodzenia się. W każdym, niezbyt dużym, obszarze przestrzeni można uważać, że kierunek normalny do powierzchni falowej jest kierunkiem rozchodzenia się fali. Można wprowadzić przy tym pojęcie promieni – krzywe, których styczne w każdym punkcie pokrywają się z kierunkiem rozchodzenia się fali \cite{TeoriaPolaLifszyc}.