Dla płaskiej fali monochromatycznej funkcja ma postać:
%
\begin{equation}\label{eq:eikonal}f=ae^{i(\sprod{k}{r}-\omega t + \alpha)}=ae^{i\psi}.\end{equation}
%
Gdy fala nie jest płaska i $\lambda \rightarrow 0$, amplituda $a$ jest funkcją współrzędnych i czasu, a faza,~zwana także eikonałem, nie ma tak prostej postaci jak we wzorze \ref{eq:eikonal}. W małych obszarach przestrzeni i małych przedziałach czasu eikonał można rozwinąć w szereg z dokładnością do wyrazów pierwszego rzędu \cite{TeoriaPolaLifszyc}:
%
$$\psi=\psi_{0}+\sprod{\vec{r}}{\frac{\partial \psi}{\partial \vec{r}}}+t\frac{\partial \psi}{\partial t}.$$
%
Porównując do równania fali monochromatycznej dostajemy:
%
$$\vec{k}=\frac{\partial \psi}{\partial \vec{r}}\equiv \grad{\psi},\qquad \omega=-\frac{\partial \psi}{\partial t},$$
%
co mówi nam, że falę w każdym małym obszarze przestrzeni można postrzegać jako falę płaską. W postaci czterowymiarowej równania można zapisać jako:
%
$$k_{i}=-\frac{\partial \psi}{\partial x^{i}}.$$
%
Otrzymujemy równanie eikonału, z związku $k_{i}k^{i}=0$:
$$\frac{\partial \psi}{\partial x_{i}}\frac{\partial \psi}{\partial x^{i}}=0$$.
