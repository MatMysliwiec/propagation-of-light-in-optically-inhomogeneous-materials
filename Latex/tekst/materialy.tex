\indent Materiały optycznie niejednorodne to szeroka klasa materiałów. Odgrywają one ważną rolę w~wielu dziedzinach nauki. Poprzez medycynę \cite{huggins1999physics2000}, mikroelektronikę, \cite{debieu2013structural,bovard1990rugate,kusy1976structure}, energetykę słoneczną \cite{niklasson1985noble,smith1985surface}, komunikację \cite{huggins1999physics2000}, aż po przedmioty codziennego użytku, takie jak papier \cite{missori2004optical,missori2006modifications}. W~węższym sensie, ośrodek jest optycznie niejednorodny, gdy następuje zmiana kierunku promieni świetlnych pod wpływem gradientu współczynnika załamania. Zmiana funkcji die\-lek\-trycz\-nej -- jako skalara stałej dielektrycznej na odległościach, z założenia przekraczających długość fali światła propagującego się w~materiale, powoduje zakrzywianie się toru promieni świetlnych. 
Efekty związane z gradientami na mniejszej skali, porównywalnej z długością fali, wymagałyby opisu wykraczającego poza przybliżenie optyki geometrycznej. Ponadto, w~rozpatrywanym w~niniejszej pracy opisie przy pomocy jednego skalarnego współczynnika załamania, nie brane są pod uwagę efekty związane z anizotropią lokalnych własności materiału optycznego, kiedy w~każdym punkcie materiału wyróżnione są kierunki główne wielkości dielektrycznych, mających na ogół charakter tensorowy (analogiczna różnica dotyczy 
opisu naprężeń w~materiałach - w~przypadku materiałów izotropowych wystarczy posługiwać się pojęciem skalarnego ciśnienia,  a w~przypadku ogólniejszym, anizotropowym, należy odwołać się do pojęcia symetrycznego tensora naprężeń --- tensor taki zawiera informację o~trzech różnych skalarach -- odpowiednikach ciśnienia -- oraz informację o~orientacji w~przestrzeni trzech wzajemnie prostopadłych osi, którym te trzy skalary odpowiadają, opisując naprężenia wzdłuż tych kierunków). Pamiętając o~tym jakościowym rozróżnieniu pojęcia ośrodków niejednorodnych optycznie, poprzestaniemy na opisie biegu promieni optycznych odnosząc pojęcie niejednorodności do skalarnego współczynnika załamania,~który staje się funkcją zależną od punktu. 

Najprostszym przykładem materii niejednorodnej jest ziemska atmosfera. Przez zmienne temperatury oraz wysokości, na których jesteśmy, powietrze przybiera różne gęstości, co jest przyczyną zmiany gradientu współczynnika załamania światła. Kiedy światło opuszcza próżnię i~wchodzi do nowego ośrodka, jego prędkość ulega zmniejszeniu. Współczynnik załamania światła w~nowym ośrodku jest równy stosunkowi prędkości światła w~próżni do jego prędkości w~nowym ośrodku. Zmiany współczynnika załamania powietrza powodują, że światło pochodzące od Słońca i~gwiazd,~zwłaszcza gdy znajdują się one na horyzoncie, zakrzywia się przy wchodzeniu w~atmosferę. W~ten sposób gwiazda wydaje się wyższej na niebie niż jest w~rzeczywistości oraz nadal jest widoczna po tym jak znajdzie się pod horyzontem. Miraże czy fatamorgana powstają,~gdy nad terenem tworzą się gorące i~chłodne warstwy powietrza. Widzimy obrazy rzutowane wzdłuż linii prostych w~oparciu o~kierunek, w~którym promienie faktycznie wchodzą do oka \cite{katz2002introduction}. Z punktu widzenia obserwacji astronomicznych ważne są też przypadkowe zmiany współczynnika załamania wynikające z lokalnych zmian gęstości ośrodka na drodze promienia świetlnego. Pozorne nieuporządkowane migotanie gwiazd (ogólnie obiektów widzianych jako obiekty o~niewielkiej wielkości kątowej) i~towarzyszące mu nieuporządkowane zmiany barwy gwiazd lub ich położenia na sferze niebieskiej, to efekt owych lokalnych zmian współczynnika załamania.  

\section{W komunikacji i medycynie}
 \indent Odbicie wewnętrzne odgrywa kluczową rolę w~nowoczesnej komunikacji i~nowoczesnej medycynie poprzez światłowody. Światło jest przesyłane przez szklany pręt lub włókno tak, że padając na powierzchnię graniczną pod kątem większym niż kąt krytyczny, światło na swej drodze podlega ciągłemu całkowitemu odbiciu wzdłuż linii światłowodu bez strat związanych z odbiciem. Przy użyciu nowoczesnego szkła o~bardzo dużej przejrzystości, światłowód może przenosić sygnał świetlny bezstratnie na wiele kilometrów, bez wyraźnego tłumienia. Przyczyną, dla której skuteczniej jest używać światła w~włóknach szklanych niż elektronów w~drucie miedzianym do przesyłania sygnałów, jest to, że włókno szklane może przenosić informacje w~znacznie szybciej niż drut miedziany. Dzieje się tak dlatego, że impulsy laserowe podróżujące przez szkło,~mogą być włączane i~wyłączane znacznie szybciej niż impulsy elektryczne w~przewodzie. Praktyczne ograniczenie dla drutu miedzianego jest rzędu miliona impulsów lub bitów informacji na sekundę (co odpowiada szybkości transmisji jednego megabita na sekundę). Szybkość przesyłu informacji w~komercyjnych liniach telefonicznych jest zazwyczaj znacznie wolniejsza, niewiele przekraczając 30 do 50 tysięcy bitów informacji na sekundę (co odpowiada 30 do 50 baudów). Szybkość ta jest wystarczająca do prowadzenia rozmów telefonicznych lub przesyłania dokumentów do wydrukowania,~ale zdecydowanie zbyt wolna dla cyfrowych sygnałów telewizyjnych. Telewizja cyfrowa o~wysokiej rozdzielczości wymaga przesyłania informacji z szybkością około 3 milionów bitów lub impulsów co $3.3$ ms, co daje szybkość 90 milionów baudów. Kable światłowodowe są w~stanie przenosić impulsy lub bity w~tempie około miliarda na sekundę, a zatem dobrze nadają się do przesyłania zdjęć, utrzymania wielu rozmów telefonicznych jednocześnie czy telewizji cyfrowej. Łącząc wiele cienkich włókien razem, można przesłać kompletny obraz wzdłuż wiązki. Jeden koniec wiązki umieszcza się przy obserwowanym obiekcie, a jeśli włókna nie są pomieszane, obraz pojawia się na drugim końcu. Aby przekazać obraz o~wysokiej rozdzielczości, potrzebna jest wiązka około miliona włókien. Do wytworzenia takiej wiązki podgrzewa się dużą grupę małych włókien szklanych, aby zmiękczyć szkło, a następnie rozciąga się pęk, aż poszczególne pasma są bardzo cienkie. Jak wspomniałem, światłowody znajdują zastosowanie w~medycynie, a w~szczególności w~diagnostyce obrazowej. Robi się to za pomocą elastycznego instrumentu światłowodowego zwanego fiberoskopem. Operacja, taka jak usunięcie pęcherzyka żółciowego, która kiedyś wymagała otwarcia powłok brzusznych i~długiego okresu rekonwalescencji, może być teraz wykonana przez mały otwór w~pobliżu pępka oraz światłowodów do podglądu procedury \cite{huggins1999physics2000}.

\section{W elektronice}
\indent Materiały niejednorodne w~postaci cienkich warstw są wykorzystywane w~nowoczesnej mikroelektronice oraz przemyśle półprzewodników. Przykładem są warstwy wytworzone z niestechiometrycznego azotku krzemu \cite{debieu2013structural}. Ponadto w~przemyśle optycznym układy warstwowe składające się z cienkich warstw o~różnych współczynnikach załamania światła są czasami zastępowane przez odpowiednie niejednorodne cienkie warstwy. Jest to spowodowane lepszymi właściwościami optycznymi tych warstw w~porównaniu z układami warstwowymi, z powodu ich budowy w~szczególności chropowatości granic powierzchni. Przykładem takiej niejednorodnej cienkiej warstwy zastępującej układy warstwowe są filtry z gradientowym współczynnikiem załamania \cite{bovard1990rugate}. W układach mikroelektronicznych interesujące zastosowanie mają tak zwane grubowarstwowe rezystory cermetowe. Elementy te składają się z mieszaniny cząstek szkła i~przewodzących tlenków metali w~płynie organicznym. Najczęściej stosowanymi materiałami przewodzącymi są tlenki rutenu (RuO$_2$, Pb$_2$Ru$_2$O$_6$, Bi$_2$Ru$_2$O$_7$). Rezystory nakładane są metodą sitodruku, w~pożądany wzór, suszone i~wypalane (ogrzewanie do wysokiej temperatury). Grubość powstałych warstw wynosi zwykle 10-20 $\mu$m. Re\-zy\-sty\-wność warstw można zmieniać w~zakresie kilku rzędów wielkości poprzez zmianę ilości używanego tlenku metalu. Jest także zależne od temperatury, gdzie wykazuje minimum w~pobliżu temperatury pokojowej, co mówi o~niskim temperaturowym współczynniku rezystancji (TWR) \cite{kusy1976structure}.

\section{W energetyce słonecznej}
\indent Do efektywnego zbierania energii słonecznej niezbędne są powłoki selektywne, które charakteryzują się wysoką absorpcją energii słonecznej i~niską emisją termiczną. Stwierdzono, że bardzo dobrze nadają się do tego celu niejednorodne kompozyty metal-izolator. Współwystępujące kompozyty takie jak Ni-Al$_{2}$O, Pt-Al$_{2}$O$_{3}$, i~Co-Al$_{2}$O$_{3}$ \cite{niklasson1985noble}. Wydajność powłok można dodatkowo poprawić poprzez wykorzystanie chropowatości powierzchni. Dobrą selektywność wykazuje również kompozyt Mo-MoO$_{2}$, wytwarzany metodą chemicznego osadzania z fazy gazowej (czarny molibden). Inną komercyjną powłoką jest niklowy pigmentowany anodowy tlenek glinu wytwarzany metodą anodowania i~elektrolitycznego barwienia. W tych powłokach selektywna ab\-sor\-pcja cząstek metalu jest często wzmacniana przez inne efekty, takie jak interferencja i~tekstura powierzchni. Selektywne właściwości optyczne kompozytów metal-izolator mogą być w~dużym stopniu zrozumiane w~ramach teorii ośrodków \cite{smith1985surface}. Konieczne są dalsze badania w~celu uzyskania ilościowego zrozumienia właściwości optycznych kompozytów metal-izolator w~szerokim zakresie składu.

\section{Przykład materiału, o którym mało się mówi - papier}
\indent Arkusze papieru stanowią mało znany przykład materiałów, których właściwości optyczne są silnie regulowane przez efekty rozpraszania światła. Arkusze papieru składają się głównie z sieci włókien celulozy, których średnica waha się od około 1 do około 10 $\mu$m. Biały wygląd niezdegradowanego papieru wynika z jego efektu rozpraszania światła spowodowanego obecnością mniejszych składników ułożonych w~nieregularny sposób \cite{missori2004optical}. Papier jest materiałem o~dużym znaczeniu zarówno ze względu na jego zastosowania przemysłowe, jak i~wartość historyczną. Ma on ogromny wachlarz zastosowań, z których najbardziej powszechnym jest jako materiał do pisania. W dziedzinie przemysłu jest on również wykorzystywany jako izolator w~transformatorach elektrycznych dużej mocy. W czasach starożytnych papier otrzymywano z czystych włókien celulozowych bawełny, lnu lub konopii. Obecnie najczęściej otrzymuje się go z masy drzewnej. Papier może zawierać również dodatki, głównie środki usztywniające i~wypełniacze (takie jak skrobia, żelatyna, kalafonia, ałun, kreda), w~zależności od zastosowanej techniki produkcji. Celuloza jest liniowym homopolimerem złożonym z jednostek $\beta$-D-glukopiranozowy, połączonych ze sobą tworząc łańcuchy. Włókna powstają, ponieważ łańcuchy celulozy mają silną tendencję do agregacji w~wysoko uporządkowane jednostki strukturalne poprzez rozbudowaną sieć zarówno wewnątrz- jak i~międzycząsteczkowych wiązań wodorowych. W konsekwencji tworzy się układ hierarchiczny od pojedynczych łańcuchów aż do włókien. Podstawowe składniki supramolekularnej struktury celulozy obejmują zespół domen wysoko uporządkowanych (krystalicznych) oraz regionów nieuporządkowanych (amorficznych). Stopień degradacji papieru po pewnym czasie zależy od warunków środowiskowych, jakim został poddany. Wśród produktów utleniania, te które są odpowiedzialne za żółknięcie nazywane są chromoforami. Czysta celuloza nie absorbuje światła powyżej około 200 nm \cite{missori2006modifications}. Żółte zabarwienie widoczne w~starzejących się papierach wynika głównie z faktu, że chromofory w~papierze pochłaniają wyższe pasmo energetyczne światła widzialnego (odpowiadające fioletowi i~niebieskiemu) i~w~znacznym stopniu rozpraszają część żółtą i~czerwoną, dając w~ten sposób charakterystyczny żółto-brązowy odcień \cite{missori2004optical, missori2006modifications}.