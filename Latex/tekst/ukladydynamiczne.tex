
\indent
Każdy układ równań różniczkowych zwyczajnych, niezależnie od liczby takich równań oraz rzędu poszczególnych równań, można sprowadzić do ogólnej postaci układu dynamicznego $$Y'(x)=F(x,Y(x)),$$ gdzie $x$ jest zmienną niezależną (zazwyczaj czasem) a $Y(x)$ jest wektorem poszukiwanych zmiennych konfiguracyjnych. Funkcja wektorowa F przyporządkowuje każdej wartości $x$ wektor pochodny $Y'$ zależny od zmiennej niezależnej $x$ (na ogół od czasu $t$). $Y'(x)$ interpretujemy jako wektor prędkości poruszającego się punktu $Y(t)$, zatem powyższe równanie opisuje pewien ruch w abstrakcyjnej przestrzeni  stanów konfiguracji takiego układu. Każde rozwiązanie tego równania będzie ruchem odbywającym się pod wpływem pola $F$,~fun\-kcja wektorowa $F$ określa w tej przestrzeni ciągłe pole wektorowe (przy założeniu, że funkcja ta jest ciągła). 

Mając układ  równań różniczkowych zwyczajnych dowolnego rzędu, zawsze możemy go przekształcić do układu równań pierwszego rzędu przez wprowadzenie dodatkowych zmiennych niezależnych. W szczególności 
gdy rozwiązujemy układ mechaniczny złożony z zespołu $N$ cząstek w~przestrzeni fizycznej, których położenia opisane są równaniami Newtona drugiego rzędu,~można przepisać je do równoważnego układu 6N równań pierwszego rzędu. Na koniec otrzymujemy układ równań bezwymiarowych przez przekształcenia w takim sposób, aby wszystkie funkcję miały wspólny argument.

\subsection{Przykłady}

\subsubsection{Cząstka w jednorodnym polu elektrycznym i jednorodnym polu magnetycznym}
\indent Zakładamy, że pola $\vec{B}$ oraz $\vec{E}$ są jednorodne i wzajemnie równoległe (oba wektory są stałe w rozpatrywanym fragmencie przestrzeni i niezależne od czasu). Osie układu współrzędnych dobieramy tak, że $\vec{E}$ odpowiada kierunkowi wersora osi $z$. 
Wówczas równania ruchu dla cząstki o masie $m$ i ładunku elektrycznym $q$ w tym polu
$$m\ddot{\vec{r}}=q \vec{E}+q\dot{\vec{r}}\times\vec{B}$$ przybierają postać następujących $3$ liniowych równań różniczkowych zwyczajnych  drugiego rzędu
$$m\ddot{x}=q B \dot{y},\qquad
m\ddot{y}=-q B \dot{x}, \qquad
m\ddot{z}=q E.$$ %
Po wprowadzeniu prędkości $v_x\equiv\dot{x}$, $v_y\equiv\dot{y}$ oraz $v_z\equiv\dot{z}$, równania powyższe przybierają postać równoważną układowi $6$ liniowych równań różniczkowych zwyczajnych pierwszego rzędu
$$
\dot{v}_x=\frac{q B}{m} v_y,\qquad \dot{x}=v_x,\qquad
\dot{v}_y=-\frac{q B}{m} v_x,\qquad\dot{y}=v_y,\qquad
\dot{v}_z=\frac{q E}{m},\qquad\dot{z}=v_z.$$
Dalej, wprowadzając bezwymiarową zmienną fazową $\phi\equiv \frac{Bq}{m}t$ w miejsce czasu,  skalę długości $r_o=\frac{Em}{B^2q}$ oraz skalę prędkości $v_o=\frac{E}{B}$, otrzymuje się równowazny układ równań bezwymiarowych
$$\psi_1'=\psi_2,\qquad \psi_2'=-\psi_1,\qquad \psi_3'=1,\qquad \xi_1'=\psi_1,\qquad \xi_2'=\psi_2
,\qquad \xi_3'=\psi_3,$$ gdzie 
$x(t)\equiv r_o\,\xi_1(\phi)$, $v_x(t)\equiv v_o\,\psi_1(\phi)$,
$y(t)\equiv r_o\,\xi_2(\phi)$, $v_y(t)\equiv v_o\,\psi_2(\phi)$,
$z(t)\equiv r_o\,\xi_3(\phi)$, $v_z(t)\equiv v_o\,\psi_3(\phi)$, a argumentem wszystkich funkcji jest zmienna niezależna $\phi$. Wprowadzając teraz wektor $Y=[\xi_1,\xi_2,\xi_3,\psi_1,\psi_2,\psi_3]$, odwzorowanie $F$ odpowiadające rozważanemu zagadnieniu ruchu zadane jest układem liniowym 
$$\left[\begin{array}{c}
y'_1\\y'_2\\y'_3\\y'_4\\y'_5\\y'_6\end{array}\right]=
F(\phi,Y)\equiv\left[\begin{array}{cccccc}
0 & 0 & 0 & 1 & 0 & 0\\
0 & 0 & 0 & 0 & 1 & 0\\
0 & 0 & 0 & 0 & 0 & 1\\
0 & 0 & 0 & 0 & 1 & 0\\
0 & 0 & 0 & -1 & 0 & 0\\
0 & 0 & 0 & 0 & 0 & 0\\
\end{array}\right]\cdot \left[\begin{array}{c}
y_1\\y_2\\y_3\\y_4\\y_5\\y_6\end{array}\right]+\left[\begin{array}{c}
0\\0\\0\\0\\0\\1\end{array}\right] .$$

\subsubsection{Odwzorowanie $F(\chi,Y)$ odpowiadające zredukowanym równaniom toru promienia świetlnego}
W poprzednich rozważaniach uzyskaliśmy układ zredukowanych równań toru ruchu promienia świetlengo:
$$
\left\{\begin{array}{l}
  \dot{x}=\cos{\varphi} \\ 
  \dot{y}=\sin{\varphi} \\
  \dot{\varphi}=\nu_{,y}\cos{\varphi}-\nu_{,x}\sin{\varphi}
\end{array}\right.
$$
Dodatkowo, aby móc analizować krzywą promienia, do powyższego układu równań wprowadzamy zmienną czasową $t$ będącą funkcją długości łuku $s$ i zdefiniowaną według wzoru:
$$t-t_0=\int\limits_{s_0}^{s_1}\!\! \frac{ds}{v}=\int\limits_{s_0}^{s_1}\!\! \frac{ds}{c}n,$$
z skąd wynika równanie różniczkowe:
$$\dot{t}=\frac{dt}{ds}=\frac{n}{c},$$
gdzie $t_0$ jest chwilą początkową odpowiadającą parametrowi $s_0$.
Przyjmując bezwymiarową zmienną czasową $\tau=c t$ otrzymujemy dodatkowe równanie postaci:
$$\tau=\frac{\ud{\tau}}{\ud{s}}=n(x,y)=e^{\nu(x,y)}.$$
Aby skonstruować odpowiadające temu układowi równań różniczkowych odwzorowanie $F(\chi,Y(\chi))$,~przyjmujemy nastepujące pryporządkowania: $s\to \chi$, $x \to y_1$, $y \to y_2$, $\phi \to y_3$, $\tau \to y_4$, skąd $\dot{x} \to y'_1$, $\dot{y} \to y'_2$, $\dot{\varphi} \to y'_3$,  $\dot{\tau} \to y'_4$:\\
$$\begin{matrix}
F(\chi,Y)=F(s,y'_1,y'_2,y'_3,y'_4)=(cos(y_3),~sin(y_3),\\ 
\nu_{,y}(y_1,y_2)cos(y_3)-\nu_{,x}(y_1,y_2)sin(y_3),~e^{\nu(y_1,y_2).})
\end{matrix}$$
