Oznaczenia wielkości użytych w pracy:
\begin{itemize}
\item Pęd cząstki $\vec{p},$
\item Funkcja Hamiltona $\cal H,$
\item Potencjał skalarny pola elektromagnetycznego $\varphi,$
\item Potencjał wektorowy pola elektromagnetycznego $\vec{A},$
\item Natężenie pola elektrycznego $\vec{E},$
\item Natężenie pola magnetycznego $\vec{H},$
\item Gęstość ładunku $\varrho,$
\item Gęstość prądu $\vec{j}.$
\end{itemize}
W pracy został użyty uproszczony zapis wektorów czterowymiarowych, czyli zespół czterech wielkości, który opisuje współrzędne zdarzenia $(ct,x,y,z)$. Jego kwadrat jest dany wyrażeniem:
%
$$(A^{0})^{2}-(A^{1})^{2}-(A^{2})^{2}-(A^{3})^{2}.$$
%
Do uproszenia zapisu wprowadzono dwa ,,rodzaje"~składowych tego wektora. Wielkość $A^i$ jest nazywany kontrawariantną składową, a $A_i$ - kowariantną składową czterowektora, a postać kwadratu czterowektora wygląda:
%
$$A^{i}A_{i}=A^{0}A_{0}+A^{1}A_{1}+A^{2}A_{2}+A^{3}A_{3}.$$
%
Opisany sposób oznaczania sumy wektorów czterowymiarowych, jest za pomocą, tzn. niemych wskaźników.