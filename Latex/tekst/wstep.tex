\quad \quad Przybliżenie optyki geometrycznej jest metodą do śledzenia toru promieni świetlnych w~celu projektowania układów zwierciadeł, pryzmatów i soczewek z jednolitych elementów optycznych. W~ciągu ostatnich kilkudziesięciu lat, prowadzone są badania nad rozchodzeniem się sygnałów w~ruchomych ośrodkach. Przyczyniły się do tego z jednej strony zastosowania układów elektromagnetycznych na coraz mniejszych długościach fal, a z drugiej strony nowe elementy optyczne:~takie jak światłowody i zintegrowane elementy optyczne. Zakres optyki geometrycznej obejmuje obecnie takie pojęcia, jak optyka w~dużych rozmiarach, optyka geodezyjna oraz optyka ośrodków niejednorodnych. Wprowadzenie do tej dziedziny teorii pola elektromagnetycznego uzupełnia ją o takie efekty jak dyfrakcja, fale akustyczne oraz fale ewanescencyjne (zanikające,~powstałe w~wyniku odbicia na granicy ośrodków o różnych współczynników załamania,~która zanika ekspotencjalnie zależnie od pokonanej odległości). Procesy optyczne wprowadziły z~kolei pewne nowe twierdzenia geometryczne oraz metody, które mają zastosowanie w~całej nauce optyki geometrycznej.

Celem pracy jest przedstawienie zagadnień wariacyjnych w~ośrodkach niejednorodnych op\-ty\-cznie, wyprowadzenie równań propagacji z odpowiedniej zasady Hamiltona, ponadto przedstawienie metody całkowania Bogackiego-Shampine'a RK3(2) z adaptowanym krokiem, która posłuży do poszukiwania torów promieni świetlnych w~układach optycznych z dyspersją współczynnika załamania.